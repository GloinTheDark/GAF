\documentclass{article}
\usepackage{amsmath}
\usepackage{amssymb}
\usepackage{abstract}
\usepackage{sectsty}
\usepackage{geometry}
\usepackage{url}
\geometry{margin=1in}


\begin{document}




\section{The GAF Field Equation}

Gravitational Acceleration Field (GAF) Theory treats gravity as a symmetric tensor field $ h_{\mu\nu}(\vec{r}, t) $, sourced by the stress-energy tensor $ T_{\mu\nu} $, with propagation at light speed $ c $. 

\subsection{The Lorentz-covariant governing equation}
$$\square h_{\mu\nu} + \lambda (h^{\alpha\beta} h_{\alpha\beta}) h_{\mu\nu} = -\frac{8\pi G}{c^4} T_{\mu\nu},$$
where $ \square = \partial_\rho \partial^\rho = -\frac{1}{c^2} \partial_t^2 + \nabla^2 $ (in Minkowski metric $ \eta^{\mu\nu} = \operatorname{diag}(-1,1,1,1) $), and $\lambda \approx c^3 / (\hbar G)$ is a coupling constant for nonlinear self-interaction.  

The source $ T_{\mu\nu} $ is the stress-energy tensor, incorporating rest-mass density, momentum, and stresses, with relativistic corrections.

\subsection{Weak-field, Low-velocity limit}
In the weak-field, low-velocity limit, this reduces to a form analogous to linearized GR, with retardation $ t_r = t - \frac{r}{c} $.

For a general mass distribution, the field is obtained by integrating over the source:
$$h_{\mu\nu}(\vec{r}) \approx \frac{2G}{c^4} \int \frac{T_{\mu\nu}(\vec{r}', t_r)}{|\vec{r} - \vec{r}'|} \, d^3 \vec{r}' + \text{nonlinear and radiative terms}.$$

For a static point mass $ M $, the dominant components approximate the Schwarzschild metric perturbations:
$$h_{00} \approx \frac{2 G M}{c^2 r} \left( 1 + \frac{G M}{c^2 r} \right),$$
where higher-order terms emerge from the nonlinear self-interaction.

Radiative terms generate GW-like energy loss. 

For test particles, geodesics in the effective metric $ g_{\mu\nu} = \eta_{\mu\nu} + h_{\mu\nu} $ incorporate relativistic effects, with photons experiencing deflection via null geodesics.

\end{document}
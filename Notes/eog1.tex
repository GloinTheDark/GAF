\section{Evolution of the GAF Theory}

As a lifelong student of physics and the history of science, I have often reflected on how past thinkers unraveled the universe's mysteries. This curiosity led to the development of the Gravitational Acceleration Field (GAF) theory, a journey that began with a simple question about Newton's formulation of gravity.

While studying Newton's law of universal gravitation, \( F = G \frac{m_1 m_2}{r^2} \), I wondered why he framed gravity as a force rather than an acceleration. Many predecessors, including Galileo, had conceptualized gravity in terms of acceleration, emphasizing its mass-independent nature through experiments like those on inclined planes. A straightforward refactoring using Newton's second law, \( F = m a \), yields an acceleration-based form: the acceleration of a test mass due to a source mass \( m \) is \( a = G \frac{m}{r^2} \). Despite my research, I never uncovered Newton's precise reasoning—perhaps he aligned it more closely with his laws of motion, or it reflected the mechanistic worldview of his era. Regardless, this reframing immediately resonates with Einstein's equivalence principle, where gravitational and inertial mass are indistinguishable, making acceleration the natural primitive.

This insight prompted the next step: envisioning gravity as an acceleration field propagating at the speed of light, akin to electromagnetic fields. In flat spacetime, this introduces retardation effects, resolving Newton's action-at-a-distance. I considered Mercury's orbit around the Sun. Due to propagation delays, portions of the Sun approaching Mercury would exert a slightly weaker pull (reflecting their earlier positions), while receding portions would pull slightly stronger. Could this aberration explain Mercury's perihelion precession? Calculations showed it accounts for only part of the observed effect—the anomaly is smaller than measured, presenting the first major puzzle.

Next, I explored how massless photons would interact with this acceleration field. Since the effect is mass-independent, photons should deflect, bending light paths near massive bodies. Again, the math revealed deflection, but only half the amount observed in GR and confirmed by experiments like the 1919 solar eclipse. It was clear that additional relativistic concepts were needed.

At this stage, I modeled the field as a vector, but further research revealed limitations: vector theories fail to predict the correct gravitational wave polarizations (two tensor modes, plus and cross). Transitioning to a symmetric tensor field \( h_{\mu\nu} \), as in linearized GR, addressed this, aligning with LIGO detections.

The precession and deflection shortfalls pointed to nonlinearity. In GR, gravity "gravitates"—the field sources itself. Introducing a self-interaction term, initially with coupling \( c^4 / G \) (inspired by GR's scales), numerically fixed the discrepancies. However, the units mismatched; dimensional analysis led to \( \lambda \approx c^3 / (\hbar G) \), incorporating the Planck constant and hinting at quantum influences. Suddenly, the pieces aligned: the field equation emerged, matching GR in weak fields and approximating it in strong fields, but with finite "quasi-singularities" at Planck scales instead of true divergences.

This outcome was unintended—I had not set out to rival GR but to explore a field-based middle ground. Yet, GAF's singularity avoidance echoes Einstein's dissatisfaction with singularities in GR. As documented in his works \cite{EinsteinRosen1935}\cite{Einstein1939}\cite{EinsteinDeSitter1917}, Einstein viewed singularities as unphysical artifacts, seeking singularity-free alternatives. Would he have appreciated GAF's finite fields?

Where does this leave GAF? It mirrors GR's predictions so closely in tested regimes that falsifiability is a key question. Detectors like LIGO, the Event Horizon Telescope (EHT), its next-generation upgrade (ngEHT), or the Laser Interferometer Space Antenna (LISA) could reveal divergences, such as softer black hole shadows, modified ringdown echoes, or subtle nonlinear GW effects. Might the distinction between singularities and Planck-scale quasi-singularities be observable? Could GAF offer insights for quantum gravity pursuits, perhaps extending to a "Quantum GAF"? These remain open questions, beyond my current scope, but they underscore GAF's potential as a bridge for future exploration.